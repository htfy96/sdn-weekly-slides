\documentclass{beamer}
\usepackage[T1]{fontenc}
\usepackage[utf8]{inputenc}
\usepackage{lmodern}
\usepackage{minted}
\usepackage{multicol}
\usepackage{amsmath}

\usetheme{metropolis}


\title{Bandwidth-Aware Scheduling With SDN in Hadoop:
A New Trend for Big Data}
\author{Zheng Luo}

\begin{document}

\begin{frame}
  \titlepage
\end{frame}

\begin{frame}{Quick look}
This article:
\begin{itemize}
    \item proposed a scheduling algorithm for Hadoop based on SDN
        \begin{itemize}
            \item which takes \emph{link bandwidth} into consideration
        \end{itemize}
    \item on the assumption that
        \begin{itemize}
            \item nodes are homogenous
            \item Finish time of task is known
            \item Bandwidth is known
        \end{itemize}
    \item in order to minimize the completion time of the whole job
\end{itemize}
\end{frame}


\begin{frame}{Previous works}
    Most of them(BAR, etc.) assume that:
    \begin{itemize}
        \item A job consists of $m$ equivalent tasks
        \item There are $n$ nodes, of which some are local and some are remote. Each has different current load $L_s^{init}$
        \item On a local node, a task requires $C_{loc}$ to finish
        \item $r^\alpha$ remote tasks require $C_{rem}(r^\alpha)$ time to finish, where $C_{rem}(\cdot )$ is a monotone increasing function(network load) ($C_{rem}(r^\alpha) = r^\alpha (C_{loc} + q)$ in its experiment)
        \item Target:
            \begin{equation}
                \begin{split}
                = \min { \max_{s \in active\_servers}{finish\_time_{s}} } \\
                = \min { \max_{s \in active\_servers} {L_s^{init} + \sum_{a(t) = s}{C(t)} } }
                \end{split}
            \end{equation}
    \end{itemize}
\end{frame}
\begin{frame}{Architecture}
\begin{figure}
    \includegraphics[scale=.4]{hadoop_arch.png}
    \caption{Topology of a Hadoop cluster centrally controlled by SDN}
\end{figure}
\end{frame}

\begin{frame}{This paper's model}
\begin{itemize}
    \item Time to finish a task $j$ in remote node: $C_{loc} + Size_j / Bandwidth_{src, j}$
    \item Use time slot method to share link
\end{itemize}
\end{frame}

\begin{frame}[fragile]
\frametitle{Algorithm}
\footnotesize
\begin{minted}{ruby}
tasks.each |task|
    nd_loc = find_earlist_idle_time_point(local_rack.nodes)
    nd_all = find_earlist_idel_time_point(all.nodes)
    if nd_loc == nd_all
        assign(task, nd_loc)
    else
        delta = nd_loc.idle_point - nd_all.idle_point
        if possible_to_allocate_slots(source, nd_all,
            nd_all.idle_point,
            task.size, delta)
            allocate_time_slots(
                source, nd_all, nd_all.idle_point
            )
            assign(task, nd_all)
        else
            assign(task, nd_loc)
        end
    end
end
\end{minted}
\end{frame}

\begin{frame}{Experiment}
\begin{figure}
\centering
\begin{minipage}{.5\textwidth}
  \centering
    \includegraphics[scale=.18]{hadoop_completion_time.png}
    \caption{Job completion time for both wordcount and sort jobs}
\end{minipage}%
\begin{minipage}{.5\textwidth}
  \centering
 \includegraphics[scale=.2]{hadoop_arch.png}
 \caption{Topology of a Hadoop cluster centrally controlled by SDN}
\end{minipage}
\end{figure}
\end{frame}

\begin{frame}{Drawbacks}
    \begin{itemize}
        \item Always utilizing full residue bandwidth may not be optimal
            \begin{itemize}
                \item Use SDN Queue to limit rate? (described in analysis)
            \end{itemize}
        \item Local optimal != global optimal
            \begin{itemize}
                \item Add critical-path penalty for node selection?
                \item Randomized adjustment? (BAR algorithm)
            \end{itemize}
    \end{itemize}
\end{frame}

\begin{frame}[standout]
    Thanks for listening
    
    \scriptsize
    Bandwidth-Aware Scheduling With SDN in Hadoop: A New Trend for Big Data;
    Peng Qin, Bin Dai, Benxiong Huang, and Guan Xu;
    IEEE Systems Journal, March 2014
\end{frame}


\end{document}
