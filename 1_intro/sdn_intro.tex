\documentclass{beamer}
\usepackage[T1]{fontenc}
\usepackage[utf8]{inputenc}
\usepackage{lmodern}
\usepackage{minted}

\usetheme{metropolis}

\title{A quick look at SDN}
\author{Zheng Luo}

\begin{document}

\begin{frame}
  \titlepage
\end{frame}

\begin{frame}
    \frametitle{Table of contents}
    \tableofcontents
\end{frame}

\section{Why SDN}

\begin{frame}
    \frametitle{Traditional network}
    \begin{figure}
        \centering
        \includegraphics[scale=.3]{previous_net.png}
        \caption{Traditional network}
    \end{figure}
\end{frame}

\begin{frame}
    \frametitle{Problem 1: Flexibility}
    \begin{itemize}
        \item L2 switch forwards by MAC Address Table(target MAC, etc.)
        \item Router forwards by routing table(target IP, etc.)
        \item Firewall filters different types of packets with builtin policies
    \end{itemize}
    Questions:
    \begin{itemize}
        \item Can router restrict the speed of video streaming?
        \item Can switch perform customlized forwarding policies?
        \item Can we change the builtin policy of firewall dynamicly?
    \end{itemize}
    Answer:
    
    Hard. Because they are implemented as hardware. Those come with many functions often have high price.
\end{frame}

\begin{frame}
    \frametitle{Problem 2: distributed algorithm/locality}
    Many distributed algorithms run between neighbors.
    
    Pros:
    \begin{itemize}
        \item Resilient to network change
    \end{itemize}
    
    Cons(like many other distributed algorithms):
    \begin{itemize}
        \item Complexity
        \item Convergence speed
        \item Optimality
    \end{itemize}
    
\end{frame}

\begin{frame}
    \frametitle{Extract the complexity}
    \begin{itemize}
        \item 
            \begin{itemize}
                \item L2 switch \emph{forwards} by MAC Address Table with \emph{matching}
                \item and \emph{updates} MAC Address table with its own algorithm
            \end{itemize}
        \item
            \begin{itemize}
                \item Router \emph{forwards} by routing table \emph{matching}
                \item and \emph{updates} routing table by routing algorithm
            \end{itemize}
        \item 
            \begin{itemize}
                \item Firewall \emph{filters}( = forward to no port) by rule \emph{matching}
                \item and \emph{updates} rule by configuration
            \end{itemize}
    \end{itemize}
    Here we divide the job of network devices into two categories:
    \begin{description}
        \item[Data plane] Simple \emph{matching} and \emph{forward}
        \item[Control plane] Controls the rules that data plane use
    \end{description}
    Traditional network couples the two planes together. Thus control plane is distributed and often built with hardware.
\end{frame}

\begin{frame}
\frametitle{Initiative}
        \begin{description}
        \item[Data plane] Often common among devices.
        \item[Control plane] Quite different among devices.
        \end{description}
        
        \begin{itemize}
            \item Make specification about data plane
            \item Make control plane programmable
        \end{itemize}
        
\end{frame}

\section{What is SDN}
\begin{frame}
\frametitle{SDN Architecture}
    \begin{figure}
        \includegraphics[scale=.4]{sdn_arch.png}
        \caption{Architecture of SDN}
    \end{figure}
\end{frame}

\begin{frame}
    \frametitle{SDN vs traditional network}
    \begin{tabular}{l|c|c}
     & \textbf{SDN} & \textbf{traditional network} \\
    \hline
    \textbf{Control \& data plane} & Decoupled & Coupled \\
    \hline
    \textbf{Control Algorithm} & Centralized & Distributed \\
    \hline
    \textbf{Programmability} & Good & Poor \\
    \hline
    \textbf{Programming interface} & Open & Private 
    \end{tabular}
\end{frame}

\begin{frame}
    \frametitle{OpenFlow}
    \begin{figure}
        \includegraphics[scale=.4]{openflow_arch.png}
        \caption{Architecture of OpenFlow}
    \end{figure}
\end{frame}

\begin{frame}[fragile]
    \frametitle{Flow table on Switch(Simplified)}
\begin{minted}{python}
When receive packet:
    while True:
        flow_table = 0
        for flow_entry in flow_table with priority desc:
            if match(packet, flow_entry.match):
                run(flow_entry.action) 
                # may modify packet
                # or send to another table(flow_table=X)
                # or output, drop, ... and finish
                update(flow_entry.counter)
                update(flow_table.counter)
                break
        if no flow_entry in flow_table matches:
            run(flow_table.table_miss_entry) 
            # Usually send_to_controller/drop/next_table
    
\end{minted}
\end{frame}

\begin{frame}[fragile]
    \frametitle{Matching}
    \begin{minted}{python}
field = extract_from(packet, match.oxm_type)
if match.oxm_hasmask:
    field &= match.mask
return field == match.oxm_value
    \end{minted}
\end{frame}

\begin{frame}[fragile]
\frametitle{Actions}
\begin{description}
    \item[Output \texttt{port\_no}] Output to a normal port/in port/all ports/ or send to controller/specific table
    \item[Group \texttt{group\_id}] Process through group
    \item[Drop \texttt{}] Drop this packet
    \item[Set-Queue \texttt{queue\_id}] Send to queue
    \item[Meter \texttt{meter\_id}] Direct to meters
    \item[Set-Field \texttt{field\_type} \texttt{value}] Rewrite specific field
\end{description}
\end{frame}

\begin{frame}
\frametitle{Data \& Control Plane messaging}
Switch can tell controller:
\begin{itemize}
    \item Hello(during startup)
    \item Packet-in(when \texttt{output CONTROLLER})
    \item Flow-removed
\end{itemize}

Controller can tell switch:
\begin{itemize}
    \item Modify-state
    \item Read-state
    \item Send-packet
\end{itemize}

\begin{small}
    * Both lists are incomplete
\end{small}

\end{frame}

\section{How to implement QoS in OpenFlow}

\begin{frame}[fragile]
    \frametitle{Queues}
    \begin{quotation}
        OpenFlow 1.2+ support both minimum and maximum rates for a given queue.
    \end{quotation}
    Definition of Queue(OpenvSwitch):
\begin{minted}{bash}
ovs-vsctl set port eth0 qos=@newqos -- --id=@newqos \
 create qos type=linux-htb other-config:max-rate=5000000 \
 queues:0=@newqueue -- --id=@newqueue create queue \
 other-config:min-rate=3000000 \
 other-config:max-rate=3000000

# ---> Queue --[min: 3Mbps, max: 3Mbps]-> linux-htb
#  --[max: 5Mbps]--> eth0
\end{minted}
\end{frame}

\begin{frame}[fragile]
    \frametitle{Meters}
    \begin{quotation}
    Unlike queues, meters could be installed/removed/modified like a flow entry in meter table.
    
    Packet could be forwarded to group by \texttt{Meter meter\_id} action
    \end{quotation}
\begin{minted}{python}
When packet comes into meter m with rate r:
    for band in meter.bands:
        if r > band.rate:
            apply(band.type) # Usually DROP
            band.counter++
\end{minted}
\end{frame}

\begin{frame}
    \frametitle{Collecting information}
    \begin{itemize}
        \item Table level
        \item Flow level
        \item Port level
        \item Queue level
        \item Meter level
    \end{itemize}
\end{frame}

\begin{frame}[standout]
    Thanks for listening
\end{frame}
\end{document}
