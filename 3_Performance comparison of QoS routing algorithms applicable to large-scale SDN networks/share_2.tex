\documentclass{beamer}
\usepackage[T1]{fontenc}
\usepackage[utf8]{inputenc}
\usepackage{lmodern}
\usepackage{minted}
\usepackage{multicol}

\usetheme{metropolis}


\title{Performance comparison of QoS routing algorithms
applicable to large-scale SDN networks }
\author{Zheng Luo}

\begin{document}

\begin{frame}
  \titlepage
\end{frame}

\begin{frame}{Contents}
\begin{multicols}{2}
        \tableofcontents
\end{multicols}

\end{frame}


\begin{frame} {Quick look}   
This article:
\begin{itemize}
    \item summarized traditional QoS Architecture,
    \item analyzed algorithms with
            \begin{itemize}
                \item bandwidth guarantees
                \item bandwidth-delay constraint
            \end{itemize}
        measured by
            \begin{itemize}
                \item bandwidth rejection ratio(BRR)
                \item route length
            \end{itemize}
\end{itemize} 
\end{frame}

\section{Traditional QoS Architecture}
\begin{frame}{Traditional QoS Architecture}
Two main QoS Architectures today:
\begin{description}
    \item[IntServ] involves prior reservation of resources before sending, per-flow state
    \item[DiffServ] Mark the packets with priority at the border of network, no prior reservation, per-class state
\end{description}
\end{frame}

\section{Bandwidth-constrained routing algorithm}

\subsection{MHA: Minimum Hop Algorithm}
\begin{frame}{MHA: Minimum Hop Algorithm}

\begin{figure}
    \includegraphics[scale=.4]{raw.png}
    \caption{Current network}
\end{figure}
    
\end{frame}

\begin{frame}{MHA: Minimum Hop Algorithm}
    Consider a flow requiring 10 bandwidth:
    \begin{figure}
    \includegraphics[scale=.4]{mha.png}
    \caption{MHA Algorithm}
\end{figure}
Remove edge that cannot satisfy bandwidth requirement, then choose path with minimum hop
\end{frame}

\subsection{WSP: Widest shortest path}
\begin{frame}{WSP: Widest shortest path}
    \begin{figure}
    \includegraphics[scale=.4]{wsp.png}
    \caption{WSP Algorithm}
    \end{figure}
    Choose K-shortest paths, then select the one with the highest available bandwidth.
    
    Similar algorithm: SWP(Shortest widest path)
\end{frame}

\subsection{MIRA: Minimum interference Routing Algorithm}
\begin{frame}{MIRA: Minimum interference Routing Algorithm}
    Given \emph{multiple} Source-Destination pairs, how to route to satisfy as many pairs as possible?

Unfortunately this is a NP problem. Therefore, heuristics are used:

\begin{itemize}
    \item \emph{Critical path}
     whenever their capacity is reduced by 1 bandwidth-unit, the maxflow value of one or more the other SD pairs also reduces by 1 bandwidth-unit.
    \item Find the least critical feasible path
\end{itemize}

\end{frame}

\subsection{DORA: Dynamic Online Routing Algorithm}
\begin{frame}{DORA: Dynamic Online Routing Algorithm}
    \begin{description}
        \item[Offline Phase] PPV(Path potential values) for each link is calculated for each SD pair. 
            \begin{enumerate}
                \item Calculate the shortest disjoint paths($SDP_i$) of each SD pair $i$.
                \item For each SD pair($i$) and link $l$, $PPV_{i, l} = \sum_{j \neq i}{\mathsf{1}_{l \in SDP_j}} - \mathsf{1}_{l \in SDP_i} $
            \end{enumerate}
        \item[Online Phase] BWP(BandWidth Proportion)
        \begin{equation}
            weight_l = (1 - BWP) PPV + BWP \frac{1}{residual\_bandwidth}
        \end{equation}
        
        Then use Dijkstra algorithm to compute weight-optimized feasible path.
    \end{description}
\end{frame}

\subsection{Benchmark}
\begin{frame}{Benchmark: BRR}
    \begin{figure}
        \includegraphics[scale=.2]{bandwidth_benchmark.png}
        \caption{BRR for different bandwidth constrained algorithms, on toy network proposed in MIRA paper}
    \end{figure}
    
\end{frame}

\begin{frame}{Benchmark: Route length}
    \begin{figure}
        \includegraphics[scale=.3]{bandwidth_hop.png}
        \caption{Average route length for different bandwidth-constrained algorithms, on toy network proposed in MIRA paper}
    \end{figure}
    
\end{frame}

\section{Bandwidth-delay constained algorithms}
\subsection{MDA: Minimum Delay Algorithm}

\begin{frame}{MDA: Minimum Delay Algorithm}
Remove all edges unsatisfying bandwidth request, then find path with minimum delay by Dijkstra.

Simple. less efficent.
    
\end{frame}

\subsection{MDWCRA: Maximum Delay-Weighted Capacity Routing Algorithm}
\begin{frame}{MDWCRA: Maximum Delay-Weighted Capacity Routing Algorithm}
    \begin{itemize}
        \item Calculate the shortest feasible disjoint path of each SD pair(length: delay)
        \item Find critial links(bottleneck) set of SD: $C_{sd}$
        \item \begin{eqnarray}
            w(link) = \sum_{(s, d): links \in C_{sd}} {\alpha_{sd}}
        \end{eqnarray}
        Different weight function$\alpha$:
            \begin{itemize}
                \item $\alpha = 1$: number of SD pairs for which the link is critical
                \item $\alpha = 1 / delay_{s, d}$: critical link on high-delay SD pair should receive more concern
            \end{itemize}
    \end{itemize}
\end{frame}

\subsection{Benchmark}

\begin{frame}{Benchmark: BRR}
\begin{figure}
    \includegraphics[scale=.4]{brr.png}
    \caption{Overall BRR}
\end{figure}
    
\end{frame}

\begin{frame}{Stdev of residual bandwidth}
    \begin{figure}
    \includegraphics[scale=.3]{stdev.png}
    \caption{Standard deviation of residual bandwidth on links}
\end{figure}
\end{frame}

\begin{frame}[standout]
    Thanks for listening
    
    \begin{scriptsize}
Performance comparison of QoS routing algorithms applicable to large-scale SDN networks; Slavica Tomovic, Igor Radusinovic, Neeli Prasad;  EUROCON 2015 - International Conference on Computer as a Tool (EUROCON), IEEE

\end{scriptsize}
\end{frame}
\end{document}
